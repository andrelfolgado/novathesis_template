%!TEX root = ../template.tex
%%%%%%%%%%%%%%%%%%%%%%%%%%%%%%%%%%%%%%%%%%%%%%%%%%%%%%%%%%%%%%%%%%%%
%% my-chapter1.tex
%% NOVA thesis document file
%%
%% Chapter with the template manual
%%%%%%%%%%%%%%%%%%%%%%%%%%%%%%%%%%%%%%%%%%%%%%%%%%%%%%%%%%%%%%%%%%%%

\typeout{NT FILE my-chapter1.tex}%

\chapter{Introduction}
\label{ch:introduction}

\glsresetall


  The growing share of renewables in the energy mix is introducing significant challenges for market participants
  along the value chain in power markets\cite{hain_managing_2018}.
  This thesis plans to study, from an energy management point of view, a renewables portfolio,
  focusing on the stochastic production risks and the evaluation of hedging strategies
  to reduce market exposer and increase predictability and steady revenue streams.
  The performance of existing derivatives as hedges and explores the potential of quantity-related weather contracts
  proposed by major energy exchanges will be a critical focus of this thesis.
  Additionally, it will examine the evolving role of price-related derivatives as key hedging instruments,
  considering the significant influence of renewables on wholesale market prices.\\

  The increasing proportion of renewable sources in energy production is leading to heightened volatility
  and creating doubts about the sustainability of consistent and steady revenue streams.
  This is due to the fact that renewable energy sources, like wind and solar power generation,
  are reliant on weather conditions and therefore do not serve as load-serving utilities.
  As a result, electricity production from these sources needs
  to be balanced by storage technologies to provide flexibility to the system and enhance reliability.
  Long-term energy storage is made mainly possible by pumped hydroelectric power plants.
  Furthermore, these are limited as they require specific geographical conditions,
  and most of the viable available sites have already been used.
  Short-term energy storage is made possible by batteries, which are still expensive and have limited capacity.
  Nonetheless, the cost of batteries is expected to decrease significantly in the coming
  years\cite{viswanathan_2022_2022, cole_cost_2023}.
  The rise of renewable energy presents an incredible opportunity for investors,
  businesses, and governments to drive sustainable development and combat climate change.
  However, the increasing penetration of variable renewable energy can lead to more extreme spot prices and the need
  to increase the market's resilience. \\

  In conclusion, the growing share of renewables in the energy mix is transforming the landscape of power markets,
  introducing new challenges for market participants and requiring innovative solutions to manage uncertainty and
  production risks and maintain certain revenue streams.
  This thesis aims to contribute to the understanding of these challenges and potential solutions
  to manage generation risk by hedging the financial outcome of a renewable's portfolio.


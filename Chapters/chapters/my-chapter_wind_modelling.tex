%!TEX root = ../template.tex
%%%%%%%%%%%%%%%%%%%%%%%%%%%%%%%%%%%%%%%%%%%%%%%%%%%%%%%%%%%%%%%%%%%%
%% my-chapter1.tex
%% NOVA thesis document file
%%
%% Chapter with the template manual
%%%%%%%%%%%%%%%%%%%%%%%%%%%%%%%%%%%%%%%%%%%%%%%%%%%%%%%%%%%%%%%%%%%%
%\usepackage{amsmath}
%\usepackage[table]{xcolor}
%\usepackage{xltabular}
%%\usepackage{colortbl} % Added this line
%\usepackage{booktabs}
%\usepackage{enumitem}
%\usepackage{tikz}
%\usetikzlibrary{shapes, arrows, positioning, fit}


\typeout{NT FILE my-chapter_wind_modelling.tex}%

\chapter{Wind Power Generation}
\label{ch:wind_power_generation}

    This thesis aims to study the volumetric risk effect induced by the intermittent nature of wind power generation,
    required to satisfy at a given time.
    Using whether derivatives is possible to develop hedging strategies to mitigate the volumetric risk,
    allowing to reduce the uncertainty and stabilise revenues from \gls{res}.


%\section{Wind Speed}
%    \label{sec:wind_speed}
%
%    > wind speed data using \\
%        1) MERRA-2 (https://disc.gsfc.nasa.gov/) or \\
%        2) ERA5 (https://cds.climate.copernicus.eu)\\
%    > Calibrate model using BSTS \\
%        > Use model to simulate wind speed data and create multiple scenarios \\

\section{Wind Power}
    \label{sec:wind_power}

    The generation of wind power is intrinsically dependent on the velocity of wind.
    The symbiotic relationship between wind speed and the consequent energy production can be represented through the
    utilisation of turbine power curves.
    These curves provide a quantitative method
    to determine the power output of a wind turbine corresponding to a particular wind speed.

    Crucially, the power curve is associated with two significant thresholds of wind speed.
    The thresholds are defined using the Betz law, which indicates the maximum power that can be extracted from the wind,
    independent of the design of a wind turbine in open flow.
    The first one is identified as the cut-in value, below which the turbine blades fail to generate power.
    This is due to the insufficient kinetic energy present in the wind to instigate their motion.
    Conversely, exceeding the second threshold, known as the cut-out value,
    leads to the safeguarding mechanism activating to prevent any potential structural damage to the system.

    In practice, turbine manufactors provide a table of values for the power output at discrete wind speeds.
    Within the range of these two critical thresholds,
    the relationship between wind speed and the generated energy can be represented by a polynomial relationship
    interpolating manufactor data for each turbine model.
    See below~\ref{eq:power_curve}, example for a 2MW turbine.

    \begin{equation}
        PC(x) =
        \begin{cases}
            0                             & \text{if } 0 <  x < 4 \\
            21.78 x^2 - 147.96 x + 243.42 & \text{if } 4 \leq x \leq 13 \\
            2000                          & \text{if } 13 < x \leq 25 \\
            0                             & \text{if } x > 25
        \end{cases}
    \label{eq:power_curve}
    \end{equation}

    To properly model power generation, the wind speed data, usually taken at reference height,
    must be adjusted to the given hub operation level of the turbine.
    The physical law that permits the conversion of wind intensity with respect to the altitude is:

    \begin{equation}
        v_h= v_{h_0}  \cdot \left(\frac{h}{h_0}\right)^\theta \text { with } \theta=\left(\ln \frac{h}{z_0}\right)^{-1}
    \label{eq:wind_speed_height}
    \end{equation}

    Where $v_{h}$ represents the wind speed measured at the height $h$ of the wind turbine hub
    and $v_{h_0}$ is the known value of the wind speed at the specified height data is recorded.
    Additionally, the parameter $z0$ is intimately connected to the site's morphological aspects wherein our
    postulated wind turbine is positioned.
    For instance, in the absence of any physical structures like buildings or trees,
    the value of this parameter typically falls within the range of 0.01 to 0.001.
    For situations involving offshore installations, the parameter is fixed at 0.0001, whereas
    an average value of $z0 = 0.005$ is employed for an onshore implementation.

    As described, seven parameters are required to fully define the power curve of a wind turbine.

    \enumerate{
        \item The cut-in wind speed, $v_{cut-in}$, below which the turbine is unable to generate power.
        \item The cut-out wind speed, $v_{cut-out}$, above which the turbine is unable to generate power.
        \item The rated wind speed, $v_{rated}$, at which the turbine is able to generate its maximum power.
        \item The rated power, $P_{rated}$, which is the maximum power output of the turbine.
        \item The power curve provided by the manufactor, which will determine the production between the threshold.
        \item The hub height, $h$, at which the wind power is generated.
        \item The spacial factor to diferentiate turbine surroundings.
    }


\section{Wind Generation Modelling}
    \label{sec:wind_generation_modelling}

    The focus of this section is to model wind power generation based on historical real and forecasted data.
    Conventional assumptions, like data independence and identically distribution, fall short when handling time-stamped
    data streams influenced by multifaceted predictors or features.
    This challenge necessitates the application of time series models to capture the inherent temporal dependencies,
    such as those stemming from macroeconomic indicators, enterprise operation, market dynamics and weather conditions.\\

    One interest of this thesis is to explore recent techniques that work well for feature selection problems in time series
    applications. A technique that can join Bayesian inference with time series models is the~\gls{bsts}.
    Initially introduced and further explored by Scott and Varian~\cite{scott_predicting_2013, scott_bayesian_2013},
    the~\gls{bsts} model is a powerful tool for feature selection, time series forecasting and inferring causal relationships. \\

    Structural time series models are a class of state space models that decompose a time series into several components of interest.
    They have a considerable intuitive appeal, particularly for economic and social times.
    Furthermore, they provide a clear link with regressions models, both in their technical formulation and in the model selection
    methodology that they employ~\cite{harvey_forecasting_1990}. \\

    The key to handling structural time series models is the state space representation, with the state of the system
    representing the unobserved components of the time series, such as trends and seasonal effects.
    Once in the state space, the Kalman filter plays a fundamental role as it recursively computes the predictive distribution
    of the state variables, given the observations up to that point in time.

    Initially introduced by Jammalamadaka et al.~\cite{qiu_multivariate_2018}, the~\gls{bsts} can be extended to a multivariate
    time series model, which is particularly useful for the analysis of multiple time series that are related to each other.
    The~\gls{mbsts} can be used to explicitly model the correlations between different asset returns in a portfolio through
    the covariance structure specified by $\Sigma_{t}$, see equation~\ref{eq:mbsts_obs}.

\subsection{The \gls{mbsts} model}
    \label{sec:mbsts}

    Models of structural time series, exemplified by the~\gls{mbsts},
    are categorized under state space models tailored for time series data. They are represented by the ensuing equations:

    \begin{align}
        y_{t} &= Z_{t} \alpha_{t} + \varepsilon_{t}, \qquad \varepsilon_{t} \sim N_{m}(0, \Sigma_{t})
        \label{eq:mbsts_obs}\\
        \notag \\
        \alpha_{t+1} &= T_{t} \alpha_{t} + R_{t} \eta_{t}, \qquad \eta_{t} \sim N_{q}(0, Q_{t})
        \label{eq:mbsts_trans}
    \end{align}


    Equation~\ref{eq:mbsts_obs} is the observation equation, which relates the $m \times 1$
    vector of observations at time $t$ $y_{t}$ to the $d \times 1$ unobserved latent states $\alpha_{t}$,
    where $d$ is the total number of latent states.
    The $m \times p$ matrix of coefficients $Z_{t}$ links the unobserved factors and regressions effects
    with the observations vector $y_{t}$.
    The $m \times 1$ vector of observation disturbances $\varepsilon_{t}$ are assumed to have zero means and
    represented by a variance-covariance matrix $\Sigma_{t}$ of the order $m \times m$.\\

    The second equation~\ref{eq:mbsts_trans} is the transition equation, which describes the evolution of the state vector
    $\alpha_{t}$ over time, characterised by the $d \times d$ transition matrix $T_{t}$.
    The $d \times q$ matrix of coefficients $R_{t}$ is often referred to as the control matrix.
    The matrix $\eta_{t}$ is an $q \times 1$ error vector,
    with a state diffusion matrix $Q_{t}$ of the order $q \times q$, where $q \leq d$.
    In many standard cases, $q = d$ and $Q_{t}$ is the identity matrix $I_{d}$.
    Although matrix $Q_{t}$ can be specified freely,
    it is often composed of a selection from the $q$ columns of the identity matrix $I_{d}$~\cite{commandeur_introduction_2007}.\\

    In summary, the~\gls{mbsts} model is defined by the following set of parameters:

    \begin{itemize}
        \item $y_{t}$ ($m \times 1$): Observations at time $t$
        \item $\alpha_{t}$ ($d \times 1$): Unobserved latent states $d$
        \item $Z_{t}$ ($m \times p$): Coefficients linking the unobserved factors and regression effects
            with the observations vector $y_{t}$
        \item $\varepsilon_{t}$ ($m \times 1$): Observation disturbances, assumed to have zero means,
            represented by the variance-covariance matrix $\Sigma_{t}$
        \item $T_{t}$ ($d \times d$): Transition matrix characterizing the evolution of the state vector $\alpha_{t}$
        \item $R_{t}$ ($d \times q$): Control matrix of coefficients
        \item $\eta_{t}$ ($q \times 1$): Error vector with a state diffusion matrix $Q_{t}$ of the order $q \times q$ where $q \leq d$
        \item $\Sigma_{t}$ ($m \times m$): Covariance matrix of the order $m \times m$ representing the observation disturbances
        \item $Q_{t}$ ($q \times q$): State diffusion matrix
    \end{itemize}


    As per developed by Harvey~\cite{harvey_forecasting_1990}, a basic structural time series model can be represented
    by a direct interpretation of its underlying components.
    In a general, the model can be expressed as:

    \begin{equation}
        y_{t} = \mu_{t} + \theta{t} + \omega_{t} + \xi_{t} + \varepsilon_{t} ,
        \qquad \varepsilon_{t} \sim N_{m}(0, \Sigma_{\varepsilon}), \quad t = 1, 2, ..., n
    \label{eq:sts}
    \end{equation}


    where $y_{t}$, $\mu_{t}$, $\theta{t}$, $\omega_{t}$, $\xi_{t}$ and $\varepsilon_{t}$ are m-dimension vectors,
    representing targeted time series, linear trend, seasonal, cyclical and regression components
    with the observation disturbances respectively.
    For simplicity, $\Sigma_{\varepsilon}$ is a $m \times m$, positive definite matrix and assumed to be constant over time.

    Elucidating further on the state space form, $\alpha_{t}$ aggregates these multifaceted components,
    depicting the concealed and unobserved latent states, such that:

   \begin{equation}
        \alpha_{t} = [\mu_{t}, \theta_{t}, \omega_{t}, \xi_{t}]^T
    \label{eq:sts_latent}
    \end{equation}


    In the prevailing model, each component of the state is compiled individually,
    where every component produces an incremental influence on $y_{t}$.
    The adaptability of this model facilitates the incorporation of various model elements specific to each target series.

    % Develope the different components states

\subsection{Generatate Wind Series Forecast Scenarios}
    \label{sec:wind_forecast_scenarios}

    As conventional in Bayesian data analysis, predictions are derived from our model's posterior predictive distribution.
    This is achieved by drawing samples of model parameters and unseen or hidden states from their respective posterior distribution.
    Therefore, the posterior predictive distribution of $\hat{Y}$, representing the collection of forecasted values, can be described as:

    \begin{equation}
        p(\hat{Y} | Y) = \int p(\hat{Y} | \psi) p(\psi | Y) d\psi
    \label{eq:posterior_predictive}
    \end{equation}

    Where $Y$ is the observed data and $\psi$ is the set of all the model parameters and latent states randomly drawn from
    $p(\psi | Y)$.

    The posterior distribution of the model parameters can be trained by using \gls{mcmc} algorithms.
    Taking advantage of working in the state space methods, forecast can be obtained continuing the Kalman filter after
    the last observation in the time series.
    This is known as the one-step-ahead forecast~\cite{durbin_time_2012}. \\

    It is important to note that in this model, the predictive probability density is not contingent on parameter estimates,
    nor on the inclusion or exclusion of predictors with stationary regression coefficients these have all been fully integrated.
    The Bayesian model averaging ensures that we do not unduly commit to any specific set of covariates,
    thus helping us avoid an arbitrary selection, and neither do we settle for point estimates of their coefficients,
    thereby preventing overfitting.

    The correlations amongst multiple target series are naturally accounted for when sampling for prediction values.
    The posterior predictive density is designed as a joint distribution over all predicted target series rather than a
    mere collection of univariate distributions.
    This methodology enables to project multiple target series in a comprehensive manner rather than examining each individually,
    as disjoint segments~\cite{qiu_multivariate_2018}.

    This can be particularly beneficial when generating summary statistics, such as the mean and variance-covariance,
    derived from the joint empirical distribution of forecast values.

\section{Capacity Factor}
    \label{sec:capacity_factor}

    Within the scope of this thesis, the capacity utilisation factor, also known as the capacity factor, we be employed
    as an indicator of a power production asset's performance capabilities.
    This factor represents an asset's actual output over a determined period compared to its potential output
    if it were possible for it to operate at full capacity continuously over the same timeframe.
    A higher capacity factor effectively signifies a more efficient energy production.
    When the mechanisms of energy trading are articulated, this factor can be converted directly into projected revenue,
    thus providing a tangible correlation between an asset’s performance and its financial implications.

    As previously mentioned, in chapter~\ref{sec:wind_futures},
    the capacity factor is the metric that indexes, such as NASDAQ WIDE, adopted to model future wind contracts.
    Therefore, the portfolio capacity factor will be considered to model futures contracts and evaluate the
    performance of the hedging strategies.

    The capacity factor is calculated using the following equation:

    \begin{equation}
        cf_t =
%        \frac{\text{Energy generated at period t}}{\text{Nominal Capacity Production at time t}} =
        \frac{g_{t}}{C \cdot h_{t}}
    \label{eq:capacity_factor}
    \end{equation}

    Where $g_{t}$ is the actual energy production, $C$ is the nominal capacity production and $h_{t}$ is the
    number of hours in the period $t$.
    The nominal capacity production refers to the maximum possible energy output under ideal conditions.
    The capacity factor is a dimensionless value, usually expressed as a percentage.

    For $T$ periods defined as ($t=1,....,T$), the revenue $R$ generated by a wind farm can be calculated as:

    \begin{equation}
        R = C \left( \sum_{t=1}^{T} h_{t} cf_{t} p_{t} \right)
    \label{eq:revenue_wf}
    \end{equation}

    Where $p_{t}$ is the price of energy at time $t$.
    Energy agents typically have a multiple wind generators, where can be organised in different portfolios.
    Extending the previous definition, the revenue $R$ generated by a wind portfolio composed by N assets, indexed by
    $i= 1,2, ..., N$, where $i$ can be a generation project or location.
    Let $C_{p}$ be the portfolio's nominal production capacity,
    the revenue $R_{p}$ generated by a wind portfolio can be calculated as:

    \begin{align}
        R_{p} &= \sum_{i=1}^{N} R_{i} \\
        \Rightarrow C_{p} \left( \sum_{t=1}^{T} h_{t} cf_{tp} p_{tp} \right) &=
        \sum_{i=1}^{N} C_{i} \left(  \sum_{t=1}^{T} h_{t} cf_{it} p_{tp} \right)
    \label{eq:revenue_pf}
    \end{align}

    where $cf_{tp}$ is the portfolio's (weighted average) capacity factor at time $t$, $R_{i}$ is the revenue generated by
    the asset $i$ and $cf_{it}$ is the $i$th-asset's capacity factor at time $t$.
    Dividing both sides by the portfolio's total capacity $C_{p}$:

    \begin{align}
        \sum_{t=1}^{T} h_{t} cf_{tp} p_{tp} &=
        \sum_{i=1}^{N} \frac{C_{i}}{C_{p}} \left(  \sum_{t=1}^{T} h_{t} cf_{it} p_{tp} \right) \\
        \Rightarrow \overline{R}_{p} &= \sum_{i=1}^{N} x_{i} R_{i}
    \label{eq:revenue_std}
    \end{align}

    In this context, $\overline{R}_{p}$ denotes the revenue per megawatt (MW) of installed capacity and $x_{i}$
    elucidates the proportion of the total capacity of the portfolio denoted by asset $i$,
    which is also indicative of the weight of asset $i$ in the portfolio.
    It is essential to note that, assuming the equivalence of prices for all assets integrated within the portfolio,
    equation~\ref{eq:revenue_std} retains its validity across any given price $p_{tp}$.








%!TEX root = ../template.tex
%%%%%%%%%%%%%%%%%%%%%%%%%%%%%%%%%%%%%%%%%%%%%%%%%%%%%%%%%%%%%%%%%%%%
%% my-chapter1.tex
%% NOVA thesis document file
%%
%% Chapter with the template manual
%%%%%%%%%%%%%%%%%%%%%%%%%%%%%%%%%%%%%%%%%%%%%%%%%%%%%%%%%%%%%%%%%%%%
%\usepackage{xcolor}
\typeout{NT FILE my-chapter2.tex}%

\chapter{Research Context}
%\chapter{Weather Derivatives}
\label{ch:weather_derivatives}

\glsresetall

\section{Introduction}
\label{sec:weather_intro}

    Weather plays a crucial role in the economy, affecting various industries such as energy producers,
    distributors, retailers, agriculture, and transportation.
    Extreme weather events and temperature volatility at high latitudes are increasingly impacting supply and demand
    in businesses, making hedging strategies and weather derivatives essential for managing weather-related risks.
    The first weather transaction was made in 1997 by Enron,
    which sold a weather swap to a natural gas company, Koch Industries.
    After a long thinking process,
    they found a way to transfer the weather risk to a willing counterparty\cite{barrieu_primer_2010}.\\

    In the field of risk management, two main types of weather risk could be defined\cite{leggio_using_2007}.
    The term ``Non-catastrophic weather risk'' pertains to weather events with a high likelihood
    of occurring and resulting in limited losses.
    On the other hand, ``Catastrophic weather risk'' is defined as the risk associated with weather-related catastrophes
    characterised by low probability but substantial losses.
    Hereinafter, this thesis will focus on the non-catastrophic weather risk,
    which is what concerns most energy market application.


\section{Weather Risk}
\label{sec:weather_risk}

    Weather risks, particularly temperature and wind fluctuations, have significant impacts on operational and financial
    decisions, as well as revenues in various industries.
    Some industries, such as energy, agriculture, retailing, travel, and transportation,
    are more vulnerable to weather risks, which can affect both revenues and costs or both.
    In the table~\ref{tab:weather_risks_tab}, it is possible to see the main weather risks for some important industries.
    As for the focal point of this thesis, the energy producers can be affected by temperature,
    wind, and solar irradiance and precipitation\cite{cui_applications_2015}.
    Weather risks can also affect the financial markets, as they can lead to increased volatility in commodity prices.
    \\
    \bgroup
        \rowcolors{1}{}{GhostWhite}
    %    \begin{xltabular}{\textwidth}{>{\ttfamily}lX>{\ttfamily}l}
        \begin{xltabular}{\textwidth}
            {   % This is the column specification for the table
                >{\ttfamily\centering\arraybackslash}m{0.33\textwidth}
                >{\ttfamily\centering\arraybackslash}m{0.33\textwidth}
                >{\ttfamily\centering\arraybackslash}m{0.33\textwidth}
            }
            \toprule
            \rowcolor{Gainsboro}% % bold font
                \texttt{Industry} & \texttt{Weather Type} & \texttt{Risk}  \\
            \midrule
                Energy Consumer & Temperature & Excessive or reduced demand\\
                Energy Producer & Temperature, Wind, Solar irradiance, Precipitation & Excessive or reduced demand\\
                Agriculture     & Temperature, Precipitation & Crop yield, handling, storage, pests\\
                Retailing       & Temperature, Precipitation & Reduced demand\\
                Travel          & Temperature, Precipitation, Snowfall & Cancellations or lower revenue\\
    %            Transportation  & Temperature, Precipitation, Wind, Frost day, Snowfall &
    %                Delays, Cancellations, Higher operational costs, Lower consumption\\
    %            Government      & Temperature, Precipitation, Snowfall & Higher budget costs\\
                Construction    & Temperature, Precipitation, Snowfall & Delays, Higher budget costs\\
            \bottomrule
            \caption{Weather Risks}
            \label{tab:weather_risks_tab}
        \end{xltabular}
    \egroup

\section{Weather Derivatives}
\label{sec:weather_derivatives}

    Weather derivatives present various advantages over alternative weather risk management tools
    by efficiently transferring weather-related risks to parties capable of managing or utilising them effectively.
    These instruments have underlying variables like rainfall, temperature, humidity, wind speed,
    snowfall and are similar to conventional contingent claims that depend on the price of some fundamental.
    In the energy market, particularly taking into account the focus of this work,
    they can be used as a significant alternative to insurance contracts.
    The main advantages of weather derivatives are~\cite{cui_applications_2015}:

    \begin{enumerate}
        \item Transfer weather related risks to the party which could manage or use them more eficiently;
        \item Lower contracting costs due to reduced moral hazard and adverse counterparty risks in trading
        \item Provide compensation for losses occurred;
        \item Offer a payment simply based on weather index value (the field inspection is not needed any more);
        \item Eliminate the insurable interest in the subject of insurance;
        \item More convenient means to safeguard against limited losses associated with high-probability events
        \item Be relatively easier hedged since weather risk is primarily volume risk and not price/market risk.
    \end{enumerate}

    These benefits make it natural for industries to employ weather derivatives to stabilize revenues,
    cover over-budget costs, and reimburse losses~\cite{matsumoto_simultaneous_2021}.
    Moreover, the limited correlation of underlying indices in weather derivatives with other financial indices
    positions them as an alternative asset class, contributing to portfolio diversification.
    The ability of weather derivatives to serve as both risk management tools and alternative assets underscores their
    versatility and relevance in diverse financial contexts.\\

    The market for weather derivatives is expanding, where the energy sector has been a primary driver of the growth,
    with temperature-dependent derivatives being the most prevalent~\cite{ali_pricing_2023}.
    With the increasing penetration of renewable energy sources in the energy mix becoming the main source of energy,
    the role of weather derivatives in hedging against weather-related risks will become increasingly important.

\section{Renewable Generation Costs}
\label{sec:wind_solar_lcoe}

    The analysis of the levelized cost of electricity (LCOE) for various power generation technologies in Europe in 2023
    indicates that onshore wind and utility-scale solar are the cheapest technologies to deploy.
    In 2023, onshore wind and PV have average LCOEs of €63/MWh and €51/MWh, respectively.
    These values are projected to more than halve in real terms by 2050 to €28/MWh and €24/MWh,
    making them the most cost-competitive options in the long term~\cite{abdullah_europe_2023}.
    The report also suggests that carbon-capture thermal generation,
    specifically CCGT with Carbon Capture and Storage (CCS), is the cheapest non-nuclear, zero-carbon thermal generation.
    However, onshore wind is the most appealing technology in terms of wholesale revenue and LCOE in all 15 markets analysed.
    Therefore, based on the analysis, onshore wind and utility-scale solar appear to be the best
    long-term cheapest technologies for power generation in Europe.


%\section{Applications in Solar and Wind Generation}
%\label{sec:weather_solar_wind}
%
%    Weather derivatives have been increasingly used in solar and wind energy production to hedge against
%    fluctuations in temperature and wind speed,
%    which can significantly impact energy production and revenues\cite{hain_managing_2018}.
%    By using weather derivatives, renewable energy producers can manage the financial risks associated
%    with variable weather conditions, ensuring more stable revenues and reducing the overall risk of their businesses.
%    The use of weather derivatives in solar and wind energy production has been growing,
%    with the aim of stabilising revenues, covering over-budget costs, and reimburseing losses\cite{hain_managing_2018}.
%    As the use of renewable energy sources continues to grow, the role of weather derivatives in hedging against
%    weather-related risks will become increasingly important.
%    As the underlying indices of weather derivatives have limited correlation with other financial indices.
%    In addition, they can also be used as an alternative asset class and diversify investment portfolios.
%    In the energy market, temperature futures can be used to hedge the volume risks, providing a dynamic
%    hedging strategy for energy futures using temperature futures

\section{Conclusion}
\label{sec:weather_conclusion}

    In conclusion, weather related derivatives are being employed more frequently in alternative risk portfolios with
    multiple asset classes, particularly in the energy sector.
    By using weather derivatives, renewable energy producers can better manage weather-related risks,
    ensuring more stable revenues and reducing the overall risk of their businesses.
    As the use of renewable energy sources continues to grow,
    the role of weather derivatives in hedging against weather-related risks will become increasingly important.





